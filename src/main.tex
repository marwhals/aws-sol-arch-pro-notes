%! Author = Marjan
%! Date = 15/04/2025

% Preamble
\documentclass[11pt]{book}

% Packages
\usepackage{amsmath}    % For math symbols and equations
\usepackage{graphicx}   % For including graphics
\usepackage{geometry}   % For adjusting page layout
\usepackage{fancyhdr}   % For custom headers/footers
\usepackage{hyperref}   % For hyperlinks
\usepackage{listings}   % For code listing
\usepackage{lipsum}     % For placeholder text (to test layout)
\usepackage{helvet}
\usepackage{arydshln}
\usepackage{tikz}        % For creating diagrams and drawing
\usepackage{pgfplots}    % For plotting graphs
\usepackage{wasysym}
\usepackage{float}
\usepackage{coptic}


%\usepackage{amstex} % Helvetica as a sans-serif font alternative
\renewcommand{\rmdefault}{phv} % Set the default font family to Helvetica

% Set page margins (A4 paper)
\geometry{top=1in, bottom=1in, left=1in, right=1in}

% Set up the header
\pagestyle{fancy}
\fancyhf{}
\fancyhead[L]{AWS Solutions Architecture Notes}
\fancyhead[C]{Your Name}
\fancyhead[R]{\thepage}
% Document
\begin{document}
% Title Page
    \begin{titlepage}
        \centering
        \vspace*{2in}
        \Huge \textbf{AWS Solutions Architecture Notes}
        \vfill
        \Large Your Name
        \vfill
        \Large Date: %\today
    \end{titlepage}

    \newpage

    \tableofcontents
    \newpage


    \chapter{Course Introduction}

    \paragraph{Pre-Introduction}
    The purpose of these notes are two-fold.
    \begin{itemize}
        \item Prepare for this exam in the future.
        \item Gain a high level overview of the services offered by one of the main cloud providers.
        \item Additionally it could be useful for generating Terraform examples.
    \end{itemize}

    %Exam study strategy notes here - Targets the SAP-C02

    \chapter{Identity \& Federation}

    \section{IAM - What should you know by now}
    - Users: Long term credentials
    - Groups
    - Roles: short-term credentials, uses STS
    - EC2 Instance Roles: Uses the EC2 metadata service. One role of a time per instance
    - Service Roles: API Gateways, CodeDeploy etc
    - Cross Account roles
    - Policies
    - AWS Managed
    - Customer Managed
    - Inline Policies
    - Resource Based Policies (S3 Bucket, SQS queues, etc......)

    \subsection{IAM Policies Deep Dive}
    - Anatomy of a policy: JSON doc with Effect, Action, Resource, Conditions, Policy Variables
    - Explicit DENY has precedence over ALLOW
    - Best practice: use least privilege for maximum security
    - Access Advisor: See permissions granted and when last accessed
    - Access Analyser: Analyse resources that are shared with external entity
    - Navigate Examples at: %put link to AWS docs here

    \subsection{IAM AWS Managed Policies - Administrator Access example}

    \begin{lstlisting}[language=JSON, caption=Sample JSON Data, label=lst:json-example]
{
"Version": "2012-10-17",
"Statement": [
{
"Effect": "Allow",
"Action": "*",
"Resource": "*"
}
]
}
    \end{lstlisting}
    % Add other listings as required

    \subsection{IAM Policies Conditions}
    % Insert IAM policies conditions

    \subsection{IAM Policies Variables and Tags}
    % Insert listings here

    \subsection{IAM Roles vs Resource Based Policies}
    - Attach a policy to a resource (example: S3 bucket policy) versus attaching of a using a role as a proxy
%insert a diagram
    - When you assume a role (user, application or service), you give up your original permissions and take the permissions assigned to the role.
    - When using a resource-based policy, the principal doesn't have to give up any permissions
    - Example: User in account A needs to scan a DynamoDB table in Account A and dump it in an S3 bucket in Account B

    \subsection{IAM Permissions Boundaries}
    - IAM permission boundaries are supported for users and roles (not groups)
    - Advanced feature to use a managed policy to set the maximum permissions an IAM entity can get
    - Can be used in combinations of AWS organisations SCP

    \subsection{Use cases}

    \begin{itemize}
        \item Delegate responsibilities to non administrators within their permission boundaries, for example create create new IAM users
        \item Allow developers to self-assign policies and managed their own permissions while making sure they can't 'escalate' their privileges (i.e make themselves admin)
        \item Useful to restrict one specific user (instead of a whole account using Organisations and SCP)
    \end{itemize}

    \subsection{IAM Access Analyser}
    - Find out which resources are shared externally
    - S3 Buckets
    - IAM Roles
    - KMS Keys
    - Lambda Functions and Layers
    - SQS queues
    - Secrets Manager Secrets
    - Define Zone ofTrust = AWS Account or AWS Organisation
    - Access outside of zone trusts => findings

    \subsection{IAM Access Analyser}
    - IAM Access Analyser Policy Validation
    - Validates your policy against IAM policy grammar and best practices
    - General warnings, security warnings errors suggestions
    - Provides actionable recommendations
    - IAM Access Analyser Policy Generation
    - Generates IAM policy based on access activity
    - CloudTrail logs is reviewed to generate the policy with the fine-grained permissions and the appropriate Actions and Services
    - Reviews CloudTrail logs for up to 90 days


    \section{STS}

    \subsection{Using STS to Assume a Role}
    - Define an IAM Role within your account or cross-account
    - Define which principals can access this IAM role
    - Using AWS STS (Security Token Service) to retrieve credentials and impersonate the IAM role you have access to (AssumeRole API)
    - Temporary credentials can be valid between 15 to 12 hours.

    \subsection{Assuming a Role with STS}
    - Provide access for an IAM user in one AWS account that you own to access resources in another account that you own.
    - Provide access to IAM users in AWS accounts owned by third parties
    - Provide access for services offered by AWS to AWS resources
    - Provide access for externally authenticated users (identity federation)
    - Ability to revoke active sessions and credentials for a role (by adding a policy using a time statement - AWSRevokeOlderSessions)

    When you assume a role (user, application or service), you give up your original permissions and take the permissions assigned to the role.

    \subsection{Providing Access to an IAM User in Yours or Another AWS Account That You Own}
    - You can grant your IAM users permissions to switch to roles within your AWS account or to roles defined in other AWS accounts that you own.
    %Diagram
    - Benefits:
    - You must explicitly grant your users permission to assume the role
    - Your users must actively switch to the role using the AWS management console or assume the role using the AWS CLI or AWS API
    - You can add multi-factor authentication (MFA) protection to the role so that only users who sign in with an MFA device can assume the role.
    - Least Privilege + auditing using CloudTrail
    %Diagram

    \subsection{Providing Access to AWS Accounts Owned by Third Parties}
    - Zone of trust = accounts, organisations that you own
    - Outside Zone ofTrust = 3rd parties
    - Use IAM Access Analyser to find out which resources are exposed
    - For granting access to a thrid party:
    - The third part AWS account ID
    - An External ID (secret between you and the third party)
    - To uniquely associate with the role between you and 3rd party
    - Must be provided when defining the trust and when assuming the role
    - Must be chosen by the third party
    - Define permissions in the IAM policy

    %The confused deputy, diagram

    \subsubsection{Session Tags in STS}
    - Tags that you pass when you assume an IAM Role or federate user in STS
    - aws:PrincipalTag Condition
    - Compares the tags attached to the principal making the request with the tag you specified in the policy.
    - Example: Allow a principal to pass session tags only if the principal making the reqeust has the specified tags.
    % diagram

    \subsubsection{STS Important APIs}
    - AssumeRole: access a role within your account or cross-account
    - AssumeRoleWithSAML: return credentials for users logged with SAML
    - AssumeRoleWithWebIdentity: return creds for users logged with an IdP
    - Example providers include Amazon Cognito, Login with Amazon, Facebook, Google or any OpenID Connect-compatible identity provider.
    - AWS Recommends using Cognito
    - GetSessionToken: for MFA from a user or AWS account root user
    - GetFederationToken: obtain temporary creds for a federated user, usually a proxy app that will give the creds for a federated user, usually a proxy app that will give the creds to a distributed app inside a corporate network.

    \subsection{Identity Federation in AWS}
    - Give users outside of AWS permissions to access AWS resources in your account
    - You don't need to create IAM Users (user management is outside AWS)
    - Use cases:
    - A corporate has its own identity system (e.g Active Directory)
    - Web / Mobile application that needs access to AWS resources
    - Identity Federation can have many flavours:
    - SAML 2.0
    - Custom Identity Broker
    - Web Identity Federation With(out) Amazon Cognito
    - IAM Identity Centre

    \subsubsection{SAML 2.0 Federation}
    - Security Assertion Markup Language 2.0 (SAML 2.0)
    - Open standard used by many identity providers (e.g ADFS)
    - Supports integration with Microsoft Active Directory Federations Services (ADFS)
    - Or any SAML 2.0 - compatible IdPs with AWS
    - Access the AWS Console, AWS CLI or AWS API using temporary credentials
    - No need to create IAM Users for each of your employees
    - Need to setup a trust between AWS IAM and SAML 2.0 Identity Provider (both ways)
    - Under-the-hood: Uses the STS API AssumeRoleWithSAML
    - SAML 2.0 Federation is the "old way", IAM Identity Center Federation is the new managed and simpler way
    %Insert a URL

    \subsubsection{SAML 2.0 Federation - AWS API Access}
    % Diagram here

    \subsubsection{SAML 2.0 Federation - AWS Console Access}
    % Diagram here

    \subsubsection{SAML 2.0 Federation - Active Directory DS (ADFS)}
    % Diagram here

    \subsubsection{Custom Identity Broker Application}
    - Use only if identity provider is NOT compatible with SAML 2.0
    - The identity broker is NOT compatible with SAML 2.0
    - The identity broker must determine the appropriate IAM Role
    - Uses the STS API AssumeRole or GetFederationToken
    % Diagrams

    \subsubsection{Web Identity Federation - Without Cognito}
    - Not recommended by AWS - use cognito instead
    % Insert a diagram

    \subsubsection{Web Identity Federation - With Cognito}
    - Preferred over for Web Identity Federation
    - Create IAM Roles using Cognito with the least privilege needed
    - Build trust between the OIDC IdP and AWS
    - Cognito benefits:
    - Supports anonymous users
    - Supports MFA
    - Data Synchronisation
    - Cognito replaces a Token Vending Machine (TVM)
    %Insert a diagram

    \subsubsection{Web Identity Federation - IAM Policy}
    - After being authenticated with Web Identity Federation, you can identify the user with an IAM policy variable
    - Example:
    %Insert examples

    \subsection{AWS Directory Services (AD)?}
    - Found on any Windows Server with AD Domain Services
    - Database of objects: User Accounts, Computers, Printers, File Shares, Security Groups
    - Centralised security management, create account, assign permissions
    - Objects are organised in trees
    - A group of trees
    %diagrams

    \subsubsection{What is ADFS (AS Federation Services)?}
    - ADFS provides Single Sign-On across applications
    - SAML across 3rd party: AWS Console, Dropbox, Officer365, etc.....
    %diagrams

    \subsubsection{AWS Directory Services}
    - AWS Managed Microsoft AD
    - Create your own AD in AWS, manage users, locally supports MFA
    - Establish "trust" connection with your own on premises
    %Diagrams
    - AD Connector
    - Directory Gateway (proxy) to redirect to on-premises AD, supports MFA
    - Users are managed on the on-premise AD
    %Diagrams
    - Simple AD
    - AD-compatible managed directory on AWS
    - Cannot be joined with on-premise AD
    %Diagrams

    \subsubsection{AWS Directory Services and AWS Managed Microsoft AD}
    - Managed Service: Microsoft AD in your AWS VPC
    - EC2 Windows Instances:
    - EC2 Windows instances can join the domain and run traditional AD applications (sharepoint, etc)
    - Seamlessly Domain Join Amazon EC2 Instances from Multiple Accounts and VPCs
    - Integrations
    - RDS for SQL Server, AWS Workspaces, Quicksight.......
    - AWS SSO to provide access to third party applications
    - Standalone repository in AWS or jointed to on-premises AD
    - Multi AZ deployment of AD in 2 AZ, # of DC (Domain Controllers) can be increased for scaling
    - Automated backups
    - Automated Multi-Region replication of your directory
    % Insert Diagram

    \subsubsection{AWS Microsoft Managed AD - Integrations}
    %Add diagram

    \subsubsection{Connect to on-premise AD}
    - Ability to connect your on-premise Active Directory to AWS Managed Microsoft AD
    - Must establish a Direct Connection (DX) or VPN connection
    - Can setup three kinds of forest trust
    - One-way trust: AWS -> On-Premise
    - One-way trust: On-Premise -> AWS
    - Two-way forest: trust - AWS -> On-Premise
    - Forest trust is different than synchronisation (replication is not supported)

    \subsubsection{Solution Achitecture: Active Directory Replication}
    - You may want to create a replica of your AD on EC2 in the cloud to minimise latency of in case DX or VPN goes down
    - Establish trust between the AWS Manged Microsoft AD and EC2
    % Diagram here

    \subsubsection{AWS Directory Services AD Connector}
    - AD Connector is a directory gateway to redirect director requests to your on premises Microsoft Active Directory
    - No caching capability
    - Managed users solely on-premise, no possibility of setting up a trust
    - VPN or Direct Connect
    - Doesn't work with SQL Server, doesn't do seamless joining, can't share director.
    %Insert diagram

    \subsubsection{AWS Directory Services Simple AD}
    - Simple AD is an inexpensive Active Directory-compatible service with the common directory features
    - Supports joining EC2 instances, manage users and groups
    - Does not support MFA, RDS SQL server, AWS SSO
    - Small: 500 users, large: 5000 users
    - Powered by Samba 4, compatible with Microsoft AD
    - lower cost, low scale, basic AD compatible or LDAP compatibility
    - No trust relationship

    \subsection{AWS Orgnisations}
    %Diagram

    \subsubsection{AWS Organisations - OrgnizationAccountAccessRole}
    %Diagram
    - IAM role which grants full administrator permissions in the Member account to the Management account
    - Used to perform admin tasks in the Member accounts (e.g - creating IAM users)
    - Could be assumed by IAM users in the Management account
    - Automatically added to all new Member account created with AWS organisations
    - Must be created manually if you invite an existing Member account

    \subsubsection{Multi Account Strategies}
    - Create accounts per department, per cost centre, per dev / test / prod, based on regulatory restrictions (using SCP), for better resource isolation (ex:VPC), to have separate per-account service limits isolated account for logging.
    - Multi Account vs. On Account MultiVPC
    - Use tagging standards for billing purposes
    - Enable CloudTrail on all accounts, send logs to central S3 account
    - Send CloudWatch logs to central logging account
    - Strategy to create an account for security

    \subsubsection{Orgnisational Units (OU) - Examples}
    %Insert diagrams here

    \subsubsection{AWS Organisation - Feature Modes}
    - Consolidated billing features:
    - Consolidated Billing across all accounts - single payment method
    - Pricing benefits from aggregated usage (volume discount for EC2, S3.....)

    \subsubsection{All Features (Default)}
    - Includes consolidated billing features, SCP
    - Invited accounts must approve enabling all features
    - Ability to apply an SCP to prevent member accounts from leaving the org
    - Can't switch back to Consolidated Billing Features only

    \subsubsection{AWS Organisations - Reserved Instances}
    - For billing purposes, the consolidated billing features of AWS organisations treats all the accounts in the organisation as one account.
    - The means that all accounts in the organisation can receive the hourly cost benefit of Reserved Instances that are purchased by any other account.
    - The payer account (Management account) of an organisation can turn off Reserved Instance (RI) discount and Saving Plans discount sharing for any accounts in that organisation, including the payer account.
    - This means that RIs and Saving Plans discounts aren't shared between any accounts that have sharing turned off.
    - To share an RI or Savings Plans discount with an account, both accounts must have sharing turned on.

    \subsubsection{AWS Organisation - Moving Accounts}
    - Remove the member account from the AWS organisation
    - Send an invite to the member account from the AWS organisation
    - Accept the invite to the new Organisation from the member account.
    %Diagram

    \subsubsection{Service Control Policies (SCP)}
    - Define allowlist or blocklist IAM actions
    - Applied at the OU or Account level
    - Does not apply to the Management Account
    - SCP is applied to all the Users and Roles in the account, including Root user
    - The SCP does not affect Service-linked roles
    - Service-linked roles enable other AWS services to integrate with AWS organisations and can't be restricted by SCP's
    - SCP must have an explicit Allow from the root of each OU in the direct path to the target account (does not allow anything by default)
    - Use cases:
    - Restrict access to certain services (for example: can't use EMR)
    - Enforce PCI compliance by explicitly disabling services.

    \subsubsection{SCP Hierarchy}
    %Insert diagram
    - Management Account - Can do anything (no SCP apply)
    - Account A
    - Can do anything
    - EXCEPT S3 (explicit Deny from Sandbox OU)
    - EXCEPT EC2 (explicit deny)
    - Account B and C
    - Can do anything
    - EXCEPT S3 (explicit Deny from Sandbox OU)
    - Account D
    - Can access EC2
    - Prod OU and Account E and F

    \subsubsection{SCP Examples - Blocklist and Allowlist strategies}
    %Insert JSON examples

    \subsubsection{IAM Policy Evaluation Logic}
    %Diagram and reference to amazon docs

    \subsubsection{Restricting Tags with IAM policies}
    - You can restrict specific tags on AWS resources
    - Using the aws:TagKeys Condition Key
    - Validate the Tag Keys attached to a resource against the Tag Keys in the IAM Policy
    - Example: Allow IAM users to create EBS Volumes only if it has the "Env" and "CostCenter" Tags
    - Use either ForAllValues (must have all keys) or ForAnyValue (must have any of these keys at a minimum)

    %Add JSON tags here

    \subsubsection{Using SCP to restrict creating resources without appropriate tags}
    - Prevent IAM Users/Roles in the affected member accounts from creating resources if they don't have a specific Tag
    % Add JSON example here

    \subsubsection{AWS Organisations - Tag Policies}
    - Helps you standardise tags across resources in an AWS organisation
    - Ensure consistent tags, audit tagged resources, maintain proper resources categorisation
    - You define Tag keys and their allowed values
    - Helps with AWS Cost Allocation Tags and Attribute-based Access Control
    - Prevent any non-compliant tagging operations on specified services and resources
    - Generate a report that lists all tagged/ non compliant resources
    - Use Amazon EventBridge to monitor non-compliant tags

    \subsubsection{AWS Organisation - AI Service Opt-out Policies}
    - Certain AWS AI services may use your content for continuous improvement of Amazon AI/ML services
    - Example: Amazon Lex, Amazon Comprehend, Amazon Polly.........
    - You can opt-out of having your content stored or used by AWS AI services
    - Create an Opt-out Policy that enforces this setting across all Member accounts and AWS Regions
    - You can opt-out all AI services or selected services
    - Can be attached to Organisation Root, specific OU or individual Member account

    \subsubsection{AWS Organisations - Backup policies}
    - AWS Backup enables you to create Backup Plans that define how to backup your AWS resources
    - JSON Documents that define backup plans across an AWS Organisation
    - Gives you granular control over backup up your resources (e.g backup frequency, time window, backup region,.....)
    - Can be attached to Organisation Root, specific OU or individual Member account
    - Immutable backup plans appear in Member accounts (view only)

    \subsubsection{Using SCP to Deny a Region aws:ReqeustRegion}
    %insert JSON example

    \subsection{AWS IAM Identity Center}

    \subsubsection{AWS IAM Identity Center -successor to AWS Single Sign-On- }
    - One login (single sign-on) for all your
    - AWS accounts in AWS organisations
    - Business cloud applications (e.g, Salesforce, Box, Microsoft 365)
    - SAML2.0 enabled applications
    - EC2 windows instances

    - Identity Providers
    - Built-in identity store in IAM identity center
    - 3rd party Active Directory (AD), OneLogin, Okta

    - AWS IAM Identity Center - Login Flow

    \subsubsection{AWS IAM Identity Center}
    %Insert diagram

    \subsubsection{AWS IAM Identity Center - Fine-grained Permissions and Assignments}
    - Multi-Account Permissions
    - Manage access across AWS accounts in your AWS Organisation
    - Permission Sets - a collection of one or more IAM policies assigned to users and groups to define AWS access
    %diagrams
    - Application Assignments
    - SSO access to many SAML 2.0 business applications (Salesforce, Box, Microsoft 365)
    - Provide required URLs, certificates and metadata
    - Attribute-Based Access Control (ABAC)
    - Fine-grained permissions based on users' attributes stored in IAM identity center identity store
    - Example: Cost center, title, locale
    - Use case: Define permission once, then modify AWS access by changing the attributes

    \subsubsection{AWS Control Tower}
    - Easy way to setup and govern a secure and compliant multi-account AWS environment based on best practices
    - Benefits:
    - Automate the set up of your environment in a few clicks
    - Automate ongoing policy management using guardrails
    - Detect policy violations and remediate them
    - Monitor compliance through an interactive dashboard
    - AWS ControlTower runs on top AWS Organisations:
    - It automatically sets up AWS Organisations to organise accounts and implement SCPs (Service Control Policies)

    \subsubsection{AWS Controller Tower - Account Factory}
    - Automates account provisioning and deployments
    - Enables you to create pre-approved baselines and configuration options for AWS accounts in your organisation (e.g VPC default configuration, subnets, region, ...)
    - Uses AWS service catalog to provision new AWS accounts
    %diagram

    \subsubsection{AWS Control Tower - Detect and Remediate Policy Violations}
    - Guardrail
    - Provide ongoing governance for your Control Tower environment (AWS Accounts)
    - Preventive - using SCPs (e.g Disallow Creation of Access Keys for the Root User)
    - Detective - users AWS Config (e.g Detect Whether MFA for the Root User is Enabled)
    - Example: identify non-compliant resources (e.g, untagged resources)

    %diagram

    \subsubsection{AWS Control Tower - Guardrails Levels}
    - Mandatory
    - Automatically enabled and enforced by AWS control tower
    - Example: Disallow public Read access to the Log Archive account
    - Strongly Recommended
    - Based on AWS best practices (optional)
    - Example: Enable encryption for EBS volumes attached to EC2 instances
    - Elective
    - Commonly used by enterprises (optional)
    - Examples: Disallow delete actions without MFA in S3 buckets


    \section{AWS Resource Access Manager (RAM)}
    - Share AWS resources that you own with other AWS accounts
    - Share with any account or within your Organisation
    - Avoid resource duplication!
    - VPC Subnets
    - Allow to have all the resources launched in the same subnets
    - Must be from the same AWS organisations
    - Cannot share security groups and defaultVPC
    - Participants can manage their own resources in there
    - Participants can't view, modify, delete resources that belong to other participants or the owner
    - AWS Transit Gateway
    - Route 53 (Resolver Rules, DNS Firewall Rule Groups)
    - License Manager Configurations

    \subsubsection{AWS Resource Access Manager (RAM)}
    - Aurora DB Clusters
    - ACM Private Cerficiate Authority
    - CodeBuild Project
    - EC2 (Dedicated Hosts, Capacity Reservation)
    - AWS Glue (Catalog, Database, Table)
    - AWS Network Firewall Policies
    - AWS Resources groups
    - Systems Manager Incident Manager (Contacts, Response Plans)
    - AWS Outposts (Outpost, Site)

    \subsubsection{Resource Access Manager - VPC example}
    %Insert diagram here
    - Each account......
    - Is responsible for it own resources
    - Cannot view modify or delete other resources in other accounts
    - Network is shared so.....
    - Anything deployed in the VPC can talk to other resources in the VPC
    - Applications are accessed easily across accounts, using a private IP
    - Security groups from other accounts can be referenced for maximum security
    - Use cases
    - Applications within the same trust boundaries
    - Applications with a high degree of interconnectivity

    \subsubsection{Resource Access Manager Managed Prefix List}
    - A set of one or more CIDR blocks
    - Makes it easier to configure and maintain Security Groups and Route Tables
    - Customer-Managed Prefix List
    - Set of CIDRs that you define and manage by you
    - Can be shared with other AWS accounts or AWS Organisation
    - Modify to update many security groups at once
    - AWS-Managed Prefix List
    - Set of CIDRs for AWS services
    - You can't create, modify, share or delete them.

    \subsubsection{Resource Access Manger Route 53 Outbound Resolver}
    - Helps you scale forwarding rules to your DNS in case you have multiple accounts and VPC
    %diagram


    \section{Summary of Identity and Federation}

    \begin{itemize}
        \item Users and Accounts all in AWS
        \item AWS Organisations
        \item AWS Control Tower to setup secure and compliant multi-account AWS environment (best practices)
        \item Federation with SAML
        \item Federation without SAML with a custom IdP (GetFederationToken)
        \item IAM Identity Center to connect to multiple AWS Accounts (Organisation) and SAML apps
        \item Web Identity Federation (not recommended)
        \item Cognito for most web and mobile applications (has anonymous mode, MFA)
        \item AWS Directory Service:
        \begin{itemize}
            \item Managed Microsoft AD - standalone or setup trust AD with on-premises, has MFA, seamless joins, RDS integration
            \item AD Connector - proxy requests to on-premises
            \item Simple AD - standalone and cheap AD-compatible with no MFA, no advanced capabilities
        \end{itemize}
        \item AWS RAM to share resource (example VPC subnets)
    \end{itemize}


    \chapter{Security}

    \section{CloudTrail}
    - Provides governance, compliance and audit for your AWS Account
    - CloudTrail is enabled by default.
    - Get a history of events / API calls made within your AWS Account by;
        - Console
        - SDK
        - CLI
        - AWS Services
    - Can put logs from CloudTrail into CloudWatch Logs or S3.
    - A trail can be applied to All Regions (default) or a single Region.
    - If a resource is deleted in AWS, investigate CloudTrail first

    %% TODO Insert cloud trail diagram here

    - Management Events:
        - Operations that are performed on resources in your AWS account
        - Examples:
            - Configuring security (IAM AttachRolePolicy)
            - Configuring rules for routing data (Amazon EC2 CreateSubnet)
            - Setting up logging (AWS CloudTrail CreateTrail)
        - By default, trails are configured to log management events.
        - Can separate Read Events (that don't modify resources) from Write Events (that may modify resources)

    - Data Events:
        - By default, data events are not logged (because high volume operations)
        - Amazon S3 object-level activity (ex: GetObject, DeleteObject, PutObject): can separate Read and Write Events
        - AWS Lambda function execution activity (the Invoke API)

    - CloudTrail Insights
        - Enable CloudTrail Insights to detect unusual activity in your account:
            - Inaccurate resource provisioning
            - Hitting service limits
            - Burst of AWS IAM actions
            - Gaps in periodic maintenance activity
    - CloudTrail Insights analyses normal management events to create baseline
    - And then continuously analyses write events to detect unusual patterns
        - Anomalies appear in the CLoudTrail console
        - Event is sent to Amazon S3
        - An EventBridge event is generated (for automation needs)

    %TODO Diagram

    - CloudTrail Events Retention
        - Events are stored for 90 days in CloudTrail
        - To Keep events beyond this period, log them to S3 and use Athena

    %TODO Diagram

    \section{CloudTrail - EventBridge Integration}

    %TODO Diagram Amazon EventBridge - Intercept API calls
    %TODO Diagram Amazon EventBridge - CloudTrail

    \section{CloudTrail - SA Pro}
    %Add diagram
    S3 Enhancements:
    - Enable Versioning
    - MFA Delete Protection
    - S3 Lifecycle Policy (S3 IA, Glacier)
    - S3 Object Lock
    - SSE-S3 or SSE-KMS encryption
    - Feature to perform CloudTrail Log File Integrity validation (SHA-256 for hashing and signing)

    %TODO Diagram - CloudTrail - Solution Architecture - Multi Account, Multi Region Logging
    Observations
    - The S3 Bucket policy is necessary for cross-account delivery
    - If Account A wants to access its CloudTrail files:
        - Option 1: Create a cross-account role and assume the role
        - Option 2: edit the bucket policy

    %TODO - CloudTrail - Solition Architecture - Alert for API calls
    - Log filter metrics can be used to detect a high level of API happening
    - Ex: Count occurrences of EC2 TerminateInstnaces API
    - Ex: Count of API calls per user
    - Ex: Detect high level of Denied API calls

    %TODO - Diagram - CloudTrail - Solution Architecture: Organisational Trail
    - The Organisational Trail is created in the management account.

    CloudTrail: How to react to events the fastest?
    - Overall, CloudTrail may take up to 15 minutes to deliver events
    - EventBridge:
        - Can be triggered for any API call in CloudTrail
        - The fastest, most reactive way.
    - CloudTrail Delivery in CloudWatch Logs:
        - Events are streamed
        - Can perform a metric filter to analyse occurrences and detect anomalies
    - CloudTrail Delivery in S3
        - Events are delivered every 5 minutes
        - Possibility of analysing logs integrity, deliver cross account, long-term storage.

    \section{KMS}

    AWS KMS (Key Managment Service)
    - Anytime you hear "encryption" for an AWS service, it is most likely KMS
    - Easy way to control access to your data, AWS manages keys for us.
    - Fully integrated with IAM for authorisation
    - Seamlessly integrated into:
        - Amazon EBS: encrypt volumes
        - Amazon S3: Server-side encryption of objects
        - Amazon Redshift: Encryption of data
        - Amazon RDS: Encryption of data
        - Etc.....
    - Can also use the CLI / SDK

    KMS - KMS - Key Types
    - Symmetric (AES-256 keys)
        - First offering of KMS, single encryption key that is used to Encrypt and Decrypt
        - AWS services that are integrated with KMS use Symmetric KMS keys
        - Necessary for envelope encryption
        - You never get access to the KMS key unencrypted (must call KMS API to use)
    - Asymmetric (RSA and ECC key pairs)
        - Public (Encrypt) and Private Key (Decrypt) pair
        - Used for Encrypt/Decrypt or Sign/Verify operations
        - The public key is downloadable but you can't access the Private Key unencrypted
        - Use case: encryption outside of AWS by users who can't call the KMS API

    Types of KMS Keys
    - Customer Managed Keys
        - Create, manage and use. Can enable or disable
        - Possibility of rotation policy (new key generated every year, old key preserved)
        - Can add a Key Policy (resource policy) and audit in CloudTrail
        - Leverage for envelope encryption.
    - AWS Managed Keys
        - Used by AWS service (aws/s3, aws/ebs, aws/redshift)
        - Managed by AWS (automatically rotated every 1 year)
        - View Key Policy and audit in CloudTrail
    - AWS Owned Keys
        - Create and managed by AWS, use by some AWS services to protect your resources
        - Used in multiple AWS accounts but they are no in your AWS account
        - You can't view, use, track or audit.

    Types of KMS keys
    %TODO Insert table

    KMS Key Material Origin
    - Identifies the source of the key material in the KMS key
    - Can't be changed after creation

    - KMS(AWS_KMS) - default
        - AWS KMS created and managed the key material in its own key store

    - External (EXTERNAL)
        - You import the key material into the KMS key
        - You're responsible for securing and managing this key material outside of AWS

    - Custom Key Store (AWS_CLOUDHSM)
        - AWS KMS creates the key material in a custom key store (CloudHSM Cluster)

    KMS Key Source - Custom Key Store (CloudHSM)
    - Integrate KMS with CloudHSM cluster as a Custom Key Store
    - Key materials are stored in a CloudHSM cluster that you own and manage
    - The cryptographic operations are performed in the HSMs
    - Use cases:
        - You need direct control over the HSMs
        - KMS Keys needs to be stored in a dedicated HSMs

    KMS Key Store - External
    - Import your own key material into KMS key, Bring Your Own Key (BYOK)
    - You're responsible for key material's security, availability and durability outside of AWS
    - Supports both Symmetric and Asymmetric KMS keys
    - Can't be used with Custom Key Store (CloudHSM)
    - Manually rotate your KMS key (Automatic and On-demand Key Rotation are NOT supported)

    %TODO Diagram

    KMS Multi-Region Keys
    %TODO Diagram

    - A set of identical KMS keys in different AWS Regions that can be used interchangeably (same KMS key in multiple Regions)
    - Encrypt in one Region and decrypt in other Regions (No need to re-encrypt or making cross-Region API calls)
    - Multi-Region keys have the same key ID, key material, automatic rotation........
    - KMS - Multi-Region are NOT global (Primary + Replicas)
    - Each Multi-Region key is manged independently
    - Only one primary key at a time, can promote replicas into their own primary
    - Use cases: Disaster Recovery, Global Data Management (e.g DynamoDB, Global Tables), Active-Active Applications that span multiple Regions, Distributed Signing applicatoins.



    \section{Parameter Store}
    SSM Parameter Store
    - Secure storage for configuration and secrets
    - Optional Seamless Encryption using KMS
    - Serverless, scalable, durable easy SDK
    - Version tacking of configurations / secrets
    - Security through IAM
    - Notifications with Amazon EventBridge
    - Integration with CloudFormation

    %TODO Insert diagram

    SSM Parameter Store Hierarchy

    %TODO Insert diagram

    Standard and advanced parameter tiers

    %TODO tables

    Parameter Policies (for advanced parameters)
    - Allows to assign a TTL to a parameter (expiration date) to force updating or deleting sensitive data such as passwords.
    - Can assign multiple policies at a time.



    \section{Secrets Mangager}
    AWS Secrets Manager
    - Meant for storing secrets (e.g passwords, API keys,......)
    - Capability to force rotation of secrets every X days
    - Automate generation of secrets on rotation (uses Lambda)
    - Natively supports Amazon RDS (all supported DB engines)
    - Support other databases and services (custom Lambda function)
    - Control access to secrets using Resource-based Policy
    - Integration with other AWS services to natively pull secrets from Secrets Manager: CloudFormation, CodeBuild, ECS, EMR, Fargate, EKS, Parameter Store......
    %TODO Insert diagram

    %TODO insert code listing ---- Secrets Manager with CloudFormation

    %TODO insert diagram ---- Secrets Mangager - Sharing Across Accounts

    SSM Parameter Store vs Secrets Manager
    - Secrets Manager (dollah)
    - Automatic rotation of secrets with AWS lambda
    - Lambda function is provided for RDS, Redshift, DocumentDB
    - KMS encryption is mandatory
    - Can integrate with CloudFormation
    SSM Parameter Store (dollah)
    - Simple API
    - No secret rotation (can enable rotation using Lambda triggered by EventBridge)
    - KMS encryption is optional
    - Can integrate with CloudFormation
    - Can pull a Secrets Manager secret using the SSM parameter Store API

    SSM Parameter Store vs Secrets Manager Rotation
    %TODO diagram

    \section{RDS Security}
    - KMS encryption at rest for underlying EBS volumes / snapshots
    - Transparent Data Encryption (TDE) for Oracle and SQL Server
    - SSL Encryption to RDS is possible for all DB (in-flight)
    - IAM Authentication for MySQL, PostgreSQL and MariaDB
    - Authorisation still happens within RDS (not in IAM)
    - Can copy an un-encrypted RDS snapshot into an encrypted one
    - CloudTrail cannot be used to track queries made within RDS

    \section{SSL Encryption, SNI, MITM}
SSL/TLS - Basics
    - SSL refers to Secure Sockets Layer, used to encrypt connections
    - TLS refers to Transport Layer Security which is a newer version
    - Nowadays, TLS certifcates are mainily used but people still refer as SSL
    - Publis SSL certificates are issued by Certifcate Authorities (CA)
    - Comodo, Symantec, GoDaddy, GlobalSign, Digicert, Letencrypt, etc
    - SSL certficates have an expiration date (you set) and must be renewed

    SSL Encryption - How it works
    %TODO insert sequence diagram
    - Asymmetric Encrutpion is expensive (SSL)
    - Symmetric encryption is cheaper
    - Asymmetric handshake is used to exchange a per-client random symmetric key.
    - Possibilty of client sending an SSL certifccate as well (two-way certificate)

    SSL - Server Name Indicattion (SNI)
    %TODO insert diagram
    - SNI solves the probelm of loading multiple SSL certificates onto one web server (to server multiple websites)
    - It's a "newer" protocol and requires the client to indicate the hostname of the target server in the inisital SSL handshake
    - The server will the find the correct certificate or return the default one.

    Note:
    - Only works for ALB and NLB (newer generation), CloudFront
    - Does not work for CLB (older gen)

    SSL - Man in the middle attacks
    %TODO - add sequence diagram
    How to prevent
    1 - Don't use public-facing HTTP, use HTTPS (meaning use SSL/TLS certificates)
    2 - Use a DNS that has DNSSEC
    - To send a client to a pirate server, a DNS response needs to be "forged" by a server which intercepts them.
    - It is possible to protect your domain name by configuring DNSSEC
    - Amazon Route 53 supports DNSSEC for domain registration
    - Route 53 supports DNSSEC for DNS service as of December 2020 (using KMS)
    - You could also run a custom DNS server on Amazon EC2 for example (Bind is the most popular, dnsmasq, KnotDNS, PowerDNS)

    \section{AWS Certificate Manager - ACM}

    AWS Certificate Manager (ACM)
    - To host public SSL certificates in AWS, you can
    - Buy your own and upload them using the CLI
    - Have ACM provision and renew public SSL certificates for you (free of cost)

    - ACM loads SSL certificates on the following integrations:
    - Load Balancers (including the onces created by EB)
    - CloudFront distributions
    - APIs on API gateways

    - SSL certificates is overall a pain to manually manage, so ACM is great to leverage in your AWS infrastructure

    %TODO Diagram

    ACM - Good to know
    - Possibility of creating public certificates
    - Must verify public DNS
    - Must be issued by a trusted public certificate authority (CA)
    - Possibilty of creating private certificates
    - For you internal applications
    - You create your own private CA
    - Your applications must trust your private CA
    - Certificate renewal:
    - Automatically done if generated provisioned by ACM
    - Any manually uploaded certificates must be renewed manually and re-uploaded.
    - ACM is a regional service
    - To use with a global application (multilpe ALB for example), you need to issue an SSL certificate
    - You cannot copy certs across regions


    \section{CloudHSM}

    - KMS => AWS manages the software for encryption
    - CloudHSM -> AWS provisions encryption hardware
    - Dedicated Hardware (HSM = Hardware Security Module)
    - You manage your own encrypton keys entirely (not AWS)
    - HSM device is tamper resistant, FIPS 140-2 Level 3 compliance
    - Supports both symmetric and asymmetric encryption (SSL/TLS keys)
    - No free tier available
    - Must use the CloudHSM Client Software
    - Redshift supports CloudHSM for database encryption and key management
    - Good option to use with SSE-C encryption

    CloudHSM Diagram
    %TODO diagram
    - IAM permissions
        - CRUD an HSM Cluster
    - CloudHSM Software
        - Manage the Keys
        - Manage the Users

    CloudHSM - High Availability
    - CloudHSM clusters are spread across Multi AZ (HA)
    - Great for availability

    %TODO diagram

    CloudHSM vs KMS
    %TODO table



    \section{Solution Architecture - SSL on ELB}
    %TODO insert diagram

    %TODO insert diagram

    - You can offload SSL to CloudHSM (SSL Acceleration)
    - Supported by NGINX Apache Web servers and IIS for windows server
    - Extra security: the SSL private key never leaves the HSM device
    - Must setup a cryptographic user (CU) on the CloudHSM device

    %TODO insert diagram

    \section{S3 Security}
    - SSE-S3: encrypts S3 objects using keys handled and managed by AWS
    - SSE-KMS: leverage KMS to manage encryption keys
        - Key usage appears in CloudTrail
        - objects made public can never be read
        - On s3:PutObject make the permission kms:GenerateDataKey is allowed
    - SSE-C: when you want to manage your own encryption keys
    - Client-Side Encryption

    Glacier: all data is AES-256 encrypted, key under AWS control

    Encryption in transit (SSL/TLS)
    - Amazon S3 exposes:
        - HTTP endpoint: non encrypted
        - HTTPS endpoint: encryption

    - You're free to use the endpoint you want, but HTTPS is reccomended
    - HTTPS is mandatory for SSE-C
    - To enforce HTTPS, use a bucket policy wit aws:SecureTransport

    Events in S3 Buckets
    - S3 Access Logs:
        - Detailed records for the reqeusts that are made to a bucket
        - Might take hours to deliver
        - Might be incomplete (best effoirt)
    - S3 Event Notifications
        - Recieve notifications when certain events happen in your bucket
        - E.g: new objects created, object removal, restore objects, replication events.
        - Destinsations: SNS, SQS queue, Lambda
        - Typically delivered in seconds but can take minutes, notification for every object if versioning is enabled, else risk of one notification for two same objects write done simultaneously.
    - Trusted Advisor:
        - Check the bucket permission (is the bucket public?)
    - Amazon EventBridge
        - Need to enable CloudTrail object level logging on S3 first
        - Target can be Lambda, SQS, SNS, etc.........

    S3 Security

    - User based
        - IAM policies - which API calls shoudl be allowed for a specific user from IAM console

    - Resource Based
        - Bucket Policies - bucket wide rules from the S3 console - allows cross account
        - Object Access Contorl List (ACL) - finer grain
        - Bucket Access Contorl List (ACL) - less common

    S3 Bucket Policies
    - Use S3 bucket for policy to:
        - Grant public access to the bucket
        - Force objects to be encrypted at uplade
        - Grant access to another account (Cross Account)
    - Optional Conditions on:
        - SourceIp: Public IP or Elastic IP | VpcSourceIp: Private IP (through VPC Endpoint)
        - Source VPC or Source VPC Endpoint - only workds with VPC Endpoints
        - CloudFront Origin Identity
        - MFA
    %TODO Examples --- insert link

    S3 pre-signed URLs
    - Can generate pre-signed URLs using SDK or CLI
        - For downloads (easy can use the CLI)
        - For uploads (harder must use the SDK)
    - Valid for a default of 3600 seconds, can change timeout with --expires-in[TIME_BY_SECONDS] argument
    - Users given a pre-signed URL inherit the permissions of the person who generated the URL for GET / PUT

    Examples:
        - Allow only logged-in users to download a premium video on your S3 bucket
        - Allow an ever changing list of users to download files by generating URLs dynamically
        - Allow temporarily a user to upload a file to a precise location in our bucket

    VPC Endpoint Gateway for S3
    %TODO insert diagram

    S3 Object Lock and Glacier Vault Lock
    - S3 Object Lock
        - Adopt a WORM (Write once read many) model
        - Block an object version deletion for a specified amount of time
    - Glacier Vault Lock
        - Adopt a WORM (Write Once Read Many) model
        - Locak the policy for future edits (can no longer be changed)
        - Helpful for compliance and data retention

    \section{S3 Access Points}
    %TODO insert diagram
    - Access Points simplify security managment for S3 buckets
    - Each access point has:
        - its own DNS name (Internet Origin or VPC origin)
        - an access point policy (similar to bucket policy) - manage security at scale

    S3 - Access Points - VPC Origin
    %TODO insert diagram
    - We can define the access point ot be accessible only form within the VPC
    - You must create a VPC Endpoint to access the Access Point (Gateway or Interface Endpoint)
    - The VPC Endpoint Policy must allow access to the target bucket and Access Point

    \section{S3 Multi-Region Access Points}
    %TODO insert a diagram
    - Provide a global endpoint that spans S3 buckets in multiple AWS regions
    - Dyanamically route requests to the nearest S3 bucket (lowest latency)
    - Bi-directional S3 bucket replication rules are created to keep data in sync across regions
    - Failover Contorls - allows you to shift request across S3 buckets in different AWS regions within minutes (Active-Active or Active-Passive)

    %TODO insert a diagram

    Multi-Region Access Points - Failover Controls
    %TODO insert diagram

    \section{S3 Multi-Region Access Points - Hands On}
%TODO ------ or do via TerraForm :D

    \section{S3 Object Lambda}
    - Use AWS Lambda Functions to change the object before it is retrieved by the caller application
    - Only one S3 bucket is needed on top of whcih we create S3 Access Point and S3 Object Lambda Access Points
    - Use Cases:
        - Redacting personally indentifiable information for analytics or non production environments
        - Converting across data formats, such as converting XML to JSON
        - Resizing and watermarking images on the fly using caller-specific details, such as the user who requrest the object

    %TODO insert diagram


    \section{DDoS and AWS Shield}
    What is a DDoS
    %TODO insert a diagram

    Types of Attacks on your infrastrcuture
    - Distributed Denial os Servicve (DDoS)
        - When your service is unavailable because it is receiving too many requests
        - SYN Flood (Layer 4) - send too many TCP connection reqeusts
        - UDP reflection (Layer 4) - get other servers to send many big UDP requests
        - DNS flood attack: Overwhelm the DNS so legitimate users can't find the site
        - Slow Loris attack: a lot of HTTP connections are opened and maintained (hmmmm consider websocket servers!!)

    - Application Level Attacks:
        - more complext, more specific (HTTP level)
        - Cache bursting strategies: overload the backend database by invalidating the cache (clever)

    DDoS Protection on AWS
    - AWS Shield Standard: Protects against DDoS attack for your website and applications, for all customers at no additional cost
    - AWS Shield Advanced: 24/7 premium DDoS protection
    - AWS WAF: Filter specific requests based on rules
    - CloudFront and Route 53:
        - Availability protection using global edge network
        - Combined with AWS shield: provides DDoS attack mitigation at the edge
    - Be ready to scale - leverage AWS Auto Scaling
    - Separate static resources (S3 / CloudFront) form dynamic ones (EC2 / ALB)
    - Read the whitepaper %TODO insert link to white paper

    Sample Reference Architecture
    %TODO diagram

    AWS Shield
    - AWS Shield Standard
        - Free service that is activated for every AWS customer
        - Provides protection from attacks such SYN/UDP Floods, Reflection attacks and other layer 3/layer 4 attacks
    - AWS Shield Advanced
        - Optional DDoS mitigation service (3000 dollahs per month per organisation)
        - Protect against more sophisticated attack on Amazon EC2, Elastic Load Balancing (ELB), Amazon CloudFront, AWS Global Accelerator, Route 53
        - 24/7 access to AWS DDoS response team (DRP)
        - Protect against higher fees during usage spikes due to DDoS

    \section{AWS WAF - Web Application Firewall}
    - Protects your web applications from common web exploits (Layer 7)
    - Deploy on Application Load Balancer (localised rules)
    - Deploy on API Gateway (ruels running at the regional or edge level)
    - Deploy on CloudFront (rules on edge locations)
        - Used in front of other solutions: CLB, EC2 Instances, custom origins, S3 websites
    - Deploy on AppSync (protect your GraphQL APIs)
    - WAF is not for DDoS protection
    - Define Web ACL (Web Acess Control List)
        - Rules can include IP addresses, HTTP headers, HTTP body, or URI strings
        - Protects from common attack - SQL injection and Cross-Site scripting (XSS)
        - Size constaints, Geo Match
        - Rate-based rules (to count occurrences of events)
    - Rule Actions: Count | Allow | Block | CAPTCHA | Challenge

    AWS WAF - Mangaged Rules
    - Library of over 190 managed ruels
    - Ready to use ruels that are managed by AWS and AWS Marketplace Sellers

    - Baseline Rule Groups - general protection from common threats
        - AWSManagedRulesCommonRuleSet, AWSManagedRulesAdminProtectionRuleSet,.....
    - Use-case Specific Rule Groups - Protection for many AWS WAF use cases
        - AWSManagedRulesSQLLiRuleSet , AWSManagedRulesWindowsRulesSet, AWSMangedRulesPHPRuleSet, AWSManagedRulesWordPressRuleSet
    - IP Reputation Rule Groups - block requests based on source (e.g. malicious IPs)
        - AWSManagedRulesAmazonIpReputationList, AWSManagedRulesAnonomoyusIpList
    - Bot Control Managed Rule Group - block and manage requests from bots
        - AWSMangedRulesBotContorlRuleSet

    WAF - Web ACL - Logging
    - You can send your logs to an:
        - Amazon CloudWatch Logs log group - 5 MB per second
        - Amazon Simple Storage Service (Amazon S3) bucket - 5 minutes interval
        - Amazon Kinesis Data Firehouse - limited by Firehose quotas

    %TODO diagrams

    Solution Architecture - Enhance CloudFront Origin
    Security with AWS WAF and AWS Secrets Manager

    %TODO diagram

    \section{AWS Firewall Manager}
    - Manage rules in all accounts of an AWS organisation
    - Security policy: common set of security rules
        - WAF rules (Application Load Balancer, API Gateways, CloudFront)
        - AWS Shield Advanced (ALB, CLB, NLB, Elastic IP, CloudFront)
        - Security Groups for EC2, Application Load Balancer and ENI resources in VPC
        - AWS Network Firewall (VPC Level)
        - Amazon Route 53 Resolver DNS Firewall
        - Policies are created at the region level

    - Rules are applied to new resources as they are created (good for compliance) across all and future accounts in your Organisation

    WAF vs Firewall Manager vs Shield
    %Diagram
    - WAF, Shield and Firewall Manager are used together for comprehensive protection
    - Define your Web ACL rules in WAF
    - For granular protection of your resources, WAF along is the correct choice
    - If you want to use AWS WAF across accounts, accelerate WAF configuration, automate the protection of new resources use Firewall Manager with AWS WAF
    - Shield Advanced adds additional features on top of AWS WAF, such as dedicated support from the shield response team (SRT) and advanved reporting
    - If you're prone to frequent DDoS attacks, consider purchasing Shield Advanved


    \section{Blocking an IP Address}

    Blocking an IP address
    %TODO diagram

    Blocking an IP address - with an ALB
    %TODO diagram

    Blocking an IP address - with an NLB
    %TODO diagram

    Blocking an IP address - ALB + WAF
    %TODO diagram

    Blocking an IP address - ALB, CloudFront and WAF
    %TODO diagram

    \section{Amazon Inspector}
    Amazon Inspector
    - Automated Security Assessments
    - For EC2 instances
    - Leveraging the AWS System Manager (SSM) agent
    - Analyse against unintended network accessibility
    - Analyse the running OS against known vulnerabilities
    - For container images push to amazon ECR
    - Assessment of Container Images as they are pushed
    - For Lambda Functions
    - Identifies software vulnerabilities in function code and package dependencies
    - Assessment of functions as they are deployed

    - Reporting and integration with AWS security hub
    - Send findings to Amazon Event Bridge

    What does Amazon Inspector evaluate?
    - Remember: only for EC2 instances, Container Images and Lamabda functions
    - Continuous scanning of the infrastructure, only when needed
    - Package vulnerabilities (EC2, ECR and Lambda) - database of CVE
    - Network reachability (EC2)
    - A risk score is associated with all vulnerabilities for prioritisation

    \section{AWS Config}

    AWS COnfig
    - Help with auduting and recording compliance of your AWS resources
    - Helps records configuration and changes over time
    - AWS Conifig rules does not prevent actions from happening (no deny)
    - Questions that can be solved by AWS config
    - Is there unrestricted SSH access to my security groups?
    - Do my buckets have any public access?
    - How has my ALB configuration changed over time?
    - You can recieve alerts (SNS notifications) for any changes
    - AWS Config is a per-region service
    - Can be aggregated across regions and accounts

    AWS config resource
    - View compliance of a resource over time
    %TODO diagram

    - View configuration of a resource over time
    %TODO diagram

    - View CloudTrail API calls if enabled
    %TODO diagram

    AWS Config Rules
    - Can use AWS managed config rules (over 75)
    - Can make custom config rules (must be defined in AWS Lambda)
        - Evaluate if each EBS disk is of type gp2
        - Evaulate if each EC2 instance is t2.micro
    - Rules can be evaluated / triggered
        - For each config change
        - And / or: at regular time intervals
    - Trigger Amazon EventBridge if the rule is non-compliant (chain with Lambda)
    - Rules can have auto remediations through SSM Automations
        - If a resource is not compliant you can trigger an auto remediation
        - Ex: remediate security group rules, stop instances with non-approved tags

    \section{AWS Managed Logs}
    - Load Balancer Access Logs (ALB, NLB, CLB) -> S3
        - Access logs for your Load Balancers
    - CloudTrail Logs -> to S3 and CloudWatch logs
        - Logs for API calls made within your account
    - VPC Flow Logs -> to S3, CloudWatch Logs, Kinesis Data Firehose
        - Information about IP traffic going to and from network interfaces in your VPC
    - Route 53 Access Logs -> to CloudWatch logs
        - Log information about the queries that Route 53 receives
    - S3 Access Logs -> to S3
        - Server access logging provides detailed records for the requests that are made to a bucket
    - CloudFront Access Logs -> to S3
        - Detailed information about every user request that CloudFront receives
    - AWS Config -> to S3

    \section{Amazon GuardDuty}

    - Intelligent Threat discovery to protect your AWS Account
    - Using Machine Learning algorithms, anomaly detection, 3rd party data
    - One click to enable (30 days trial) no need to install software
    - Input data includes:
        - CloudTrail Events Logs - unusual API calls, unauthrosed deployments
            - CloudTrail Management Events - create VPC subnet, create trail
            - CloudTrail S3 Data Events - get object, list objects, delete object
        - VPC Flow logs - unusual internal traffic, unusual IP address
        - DNS Logs - compromised EC2 instances sending encoded data within DNS queries
        - Optional Feature - EKS Audit Logs, RDS and Aurora, EBS, Lambad, S3 Data Events.......
    - Can setup EventBridge rules to be notified in case of findings
    - EventBridge rules can target AWS Lambda or SNS
    - Can protect against CryptoCurrency attacks (has a dedicated "finding" for it)

    Amazon GuardDuty
    %TODO diagrams

    GuardDuty - Delegated Administrator
    - AWS organisation member accounts can be designated to be a GuardDuty Delegated Administrator
    - Have full permissions to enable and manage GuardDuty for all accounts in the Organisation
    - Can be done only using the Organisation Mangement Account
    %TODO diagrams

    \section{IAM Advanced Policies}
    IAM conditions
    %TODO - example JSON listing

    %TODO - example JSON listing

    IAM for S3
    %TODO - example JSON listing
    - s3:ListBucket permission applies to arn:aws:s3:::test -> bucket level permission

    %TODO - example JSON listing
    - s3:GetObject, s3:PutObject,s3:DeleteObject applies to arn:awn:s3:::test
    - Object level permissions

    Resource Policies and aws:PrincipalOrgID
    - aws:PrincipalOrgID can be used in any resource polices to restrict access to accounts that are memeber of an AWS organisation
    %TODO diagrram
    %TODO json listing

    \section{EC2 Instance Connect}
    EC2 Instance Connect (SendSSHPublicKey API)
    %TODO Diagram

    \section{AWS Security Hub}
    - Central security tool to manage security across several AWS accounts and automate security checks
    - Integrated dashboards showing current security and compliance status to quickly take actions
    - Automatically aggregates alerts in predefined or personal findings formates from various AWS services and AWS partner tools:
        - Config
        - GuardDuty
        - Inspector
        - Macie
        - IAM Access Analyser
        - AWS Systems Manager
        - AWS Firewal Manager
        - AWS Health
        - AWS Partner Network Solutions

    Must first enable the AWS config service

    \section{Amazon Detective}
    - GuardDuty, Macie and Security Hub are used to identify potential security issues or findings.
    - Sometimes security findings require deeper analysis to isolate the root cause and take actions - it is a complext process
    - Amazon Detective analyses, investigates and quickly identifies the root cause of security issues or suspicious activities (using ML and graphs)
    - Automatically collects and processes events from VPC Flow Logs, CloudTrail, GuardDuty and creates a unifed view.
    - Produces visualisations with details and context to get to the root cause.

    \chapter{Compute \& Load Balancing}


    \section{Solution Architecture on AWS}


    \section{EC2}


    \section{High Performance Computing (HPC)}


    \section{Auto Scaling}


    \section{Auto Scaling Update Strategies}


    \section{Spot Instances and Spot Fleet}


    \section{Amazon ECS - Elastic Container Service}


    \section{Amazon ECS - Elastic Container Registry}


    \section{Amazon EKS - Elastic Kubernetes Service}


    \section{AWS App Runner}


    \section{ECS Anywhere and EKS Anywhere}


    \section{AWS Lambda - Part 1}


    \section{AWS Lambda - Part 2}


    \section{Elastic Load Balancers - Part 1}


    \section{Elastic Load Balancers - Part 2}


    \section{API Gateway}


    \section{API Gateway - Part 2}


    \section{AWS AppSync}


    \section{Route 53 - Part 1}


    \section{Route 53 - Part 2}


    \section{Route 53 - Resolvers and Hybrid DNS}


    \section{AWS Global Accelerator}


    \section{Comparison of Solution Architecture}


    \section{AWS Outposts}


    \section{AWS WaveLength}


    \section{AWS Local Zoness}


    \chapter{Storage}


    \section{EBS and Local Instance Store}


    \section{Amazon EFS}


    \section{Amazon S3}


    \section{Amazon S3 - Storage Class Analysis}


    \section{Amazon S3 - Storage Lens}


    \section{S3 Solution Architecture}


    \section{Amazon FSx}


    \section{Amazon FSx - Solution Architecture}


    \section{AWS DataSync}


    \section{AWS DataSync - Solution Architecture}


    \section{AWS Data Exchange}


    \section{AWS Transfer Family}


    \section{AWS Storage Services Price Comparison}


    \chapter{Caching}


    \section{Cloudfront - Part 1}
    - Content Delivery Network (CDN)
    - Improves read performance, content is cached at the edge
    - 225+ Point of Presensce globally (215+ Edge Locations and 13 Regional Edge Caches)
    - Protect against Network and Application layer attacks (e.g DDoS attacks)
    - Integration with AWS Shield, AWS WAF and route 53
    - Can expose external HTTPS and can talk to internal HTTPS backends
    - Supports WebSocket protocol

    \subsection{CloudFront - Origins}
    - S3 Bucket
    - For distributing files
    - For uploading to S3 (using CloudFront as an ingres)
    - Enhanced security with CloudFront Origin Access Control (OAC)
    - MediaStore Container and MediaPackage Endpoint
    - To deliverVideo on Demand (VOD) or live streaming video using AWS Media Services
    - VPC Origin
    - For applications hosted in VPC private subnets
    - Application Load Balance / Network Load Balancer / EC2 Instances
    - Customer Origin (HTTP)
    - API Gateway (for more control.... otherwise use API Gateway Edge)
    - S3 Bucket configured as a website (enable Static Website hosting)
    - Any HTTP backend you want

    \subsection{CloudFront - S3 as an Origin}
    %Diagram

    \subsection{CloudFront vs S3 Cross Region Replication}
    - CloudFront:
    - Global Edge network
    - Files are cached for a TTL (maybe a day)
    - Great for static content that must be available everywhere

    - S3 Cross Region Replication
    - Must be setup for each region you want replication to happen
    - Files are updated in near real-time
    - Read only
    - Great for dynamic content that needs to be available at low-latency in few regions

    \subsection{CloudFront - ALB or EC2 as an origin Using VPC Origins}
    - Allows you to deliver content from you applications hosted in your VPC private subnets (no need to expose them on the Internet)
    - Deliver traffic to private
    - Application Load Balancer
    - Network Load Balancer
    - EC2 Instances

    %Diagram

    \subsection{CloudFront - EC2 or ALB as an origin}
    %Diagram

    \subsection{CloudFront - Restrict Access to Application Load Balancers and Custom Origins}
    - Prevent direct access to you ALB or Custom Origins (only access through CloudFront)
    - First, configurations CloudFront to add a CustomHTTPHeader to requests it sends to the ALB
    - Second, configure the ALB to only forward requests that contain that Customer HTTP Header
    - Keep the custom header name and value secret!

    %Diagram

    \subsection{CloudFront - Origin Groups}
    - To increase high-availability and do failover
    - Origin Group: one primary and on secondary
    - If the primary origin fail; the second on is used
    - Origins can be cross AWS regions

    %diagram


    \section{Cloudfront - Part 2}

    \subsection{CloudFront Geo Restrictions}
    - You can restrict who can access your distribution
    - Allow list: Allow you users to access your content only if they're in one of the countries on a list of approved
    - Block list: Prevent your users from accessing your content if they're in one of the countries on a blacklist of banned countries..... lol
    - The "country" is determined using a third party Geo-IP database
    - Use case: Copyright Laws to control access to content
    - Note: the geo header CloudFront-Viewer-Country is in Lambda at Edge

    \subsection{CloudFront - Pricing}
    - CloudFront Edge locations are all around the world
    - The cost of data out per edge location varies
    %Insert table

    \subsection{CloudFront - Price Classes}
    - You can reduce the number of edge locations for cost reductions
    - Three price classes
    - Price Class All: all regions - best performance
    - Price Class 200: most regions, but excludes the most expensive regions
    - Price Class 100: only the least expensive regions
    %digrams

    %prety map diagram

    \subsection{CloudFront Signed URL Diagram}
    - Signed URL with expiration to control access to content in CloudFront
    - The Signed URL are generated by an API call into CloudFront as a trusted signer
    %Diagram

    \subsection{CloudFront Signed URL vs S3 Pre-Signed URL}
    - CloudFront Signed URL:
    - Allow access to a path, no matter the origin
    - Account wide key-pair, only the root can manage
    - Can filter by IP, path, date, expiration
    - Can leverage caching features

    - S3 Pre-Signed URL:
    - Issue a request as the person who pre-signed URL
    - Uses the IAM key of the signed IAM principal
    - Limited lifetime

    %Diagram

    \subsection{CloudFront - Custom Error Pages}
    - Return an object to the viewer (e.g html) when your origin returns an HTTP 4xx or 5xx status code to CloudFront
    - Use Error Caching Minimum TTL to specify how long CloudFront caches the customer error pages

    %diagram


    \section{Lambda at Edge and CloudFront Functions}

    \subsection{CloudFront - Customisation at the edge}
    - Many modern application execute some form of the logic at the edge
    - Edge function:
    - A code that you write an attach to CloudFront distributions
    - Runs close to your users to minimise latency
    - Doesn't have any cache, only to change requests/ responses
    - CloudFront provides two types: CloudFront Functions and Lambda and Edge
    Use cases:
    - Manipulate HTTP requests and responses
    - Implement request filtering before reaching your application
    - User authentication and authorisation
    - Generate HTTP resources at the edge
    - A/B Testing
    - Bot mitigation
    - You don't have to manage any servers, deployed globally

    \subsection{CloudFront Functions and Lamabda at Edge}
    %digram

    \subsection{CloudFront - CloudFront Functions}
    - Lightweight functions written in JavaScript
    - For high-scale latency-sensitive CDN customisation
    - Sub-ms startup times, million of requests per second
    - RuN at Edge Locations
    - Processed-Based isolation
    - Used to changeViewer requests and responses:
    - Viewer Requests: after CloudFront recevives a reqeusts from a viewer
    - Viewer Response: before CloudFront forwards the response to the viewer
    - Native feature of CloudFront (manage code entirely within CloudFront)
    %diagram

    \subsection{CloudFront - Lambda at Edge}
    - Lambda functions written in NodeJS or Python
    - Scala to 1000s of reqeust/ second
    - Runs at the nearest Regional Edge Cache
    - VM-based isolation
    - Used to change CloudFront requests and responses
    - Viewer Request - after CloudFront recieves a reqeust from a viewer
    - Origin Request - before CloudFront forwards the response to the origin
    - Orign response - after CloudFront recieves the response from the origin
    - Viewer Response - before CloudFront forwards the response to the viewer
    - Author your functions in one AWS Region (some region) then CloudFront replicats to its locations

    \subsection{CloudFront Functions with Lambda at Edge}
    CloudFront Functions and Lambda at Edge can be used together
    %digram
    Note: You can't combine CloudFront Functions and Lambda at Edge in viewer events ( viewer reqeust and viewer response)

    \subsection{Using Lambda at Edge only}
    Use when you need some of the capabilites of Lambda at Edge that aren't available with CloudFront Functions (e.g longer execution time, network access,....)
    %Diagram here

    \subsection{CloudFront Funtions vs. Lamda at Edge}
    %table here

    \subsection{CloudFront Funcitons vs Lambda at Edge Use Cases}
    - CloudFront Functions
    - Cache key normalisation
    - Transform reqeust attributes (headers, cookies, query string, URL) to create an optimal Cache Key
    - Header manipulation
    - Insert/modify/ delete HTTP headers in the request or response
    - URL rewrites or redirects
    - Request authentication and authorisation
    - Create and validate user-generated tokens (e.g JWT) to allow/deny reqeusts

    Lambda at Edge
    - Longer execution time (several ms)
    - Adjustable CPU or memory
    - You code depends on a third libraries (e.g AWS SDK to access other aws Services)
    - Network access to use external services for processing
    - File system access or access to the body of HTTP requests

    \subsection{CloudFront Functions vs. Lambda at Edge - Authentication and Authorisation}
    CloudFront Functions
    %diagram

    Lambda at Edge
    %diagram

    \subsection{Lambda@Edge: Loading content based on User-Agent}
    %diagram here

    \subsection{Lambda at Edge - Global Application}
    %diagram


    \section{Lambda at Edge Reduce Latency}

    \subsection{Lambda at Edge - Route to different origin}
    %diagram


    \section{Amazon ElastiCache}

    \subsection{Amazon ElasticCache Overview}
    - The same way RDS is to get managed Relational Databases.....
    - ElastiCache is to get managed Redis or Memcached
    - Caches are in memory databases with really high performance and low latency
    - Helps reduce load off of databases for read intensive workloads
    - Helps make you application stateless
    - AWS takes care of OS maintenance / patching, optimisations, setup, configuration, monitoring failure recovery and backups.
    - \textbf{Using ElastiCache involves heavy application code changes}

    \subsection{ElastiCache Solution Architecture - DB Cache}
    - Applications queries ElastiCache, if not available, get from RDS and store in ElasticCache
    - Helps relive load in RDS
    - Cache must have an invalidation strategy to make sure only the most current data is used there.
    %diagram

    \subsection{ElastiCache Solution Architecture - User Session Store}
    - User logs into any of the application
    - The application writes the session data into ElastiCache
    - The user hits another instance of our application
    - The instance retrieves the data the user is already logged in
    %diagram

    \subsection{ElastiCache - Redis vs Memcached}
    - Redis
    - Multi AZ with Auto-Failover
    - Read Replicas to scale reads and have high availability
    - Persistent, Data Durability: Append Only File (AOF), backup and restore features
    - Memcached
    - Multi-node for partitioning of data (sharding)
    - Non Persistent
    - Backup and restore (Serverless)
    - Mult-threaded architecture


    \section{Handling Extreme Rates}
    %diagram


    \chapter{Databases}


    \section{DyanamoDB}


    \section{Amazon OpóenSearch}


    \section{RDS}


    \section{Aurora - Part 1}


    \section{Aurora - Part 2}


    \chapter{Service Communication}


    \section{Step Functions}


    \section{SQS}


    \section{Amazon MQ}


    \section{Amazon SNS}


    \section{Amazon SNS - SQS Fan Out Pattern}


    \section{Amazon SNS - Message Delivery Retries}


    \chapter{Data Engineering}


    \section{Amazon Kinesis Data Streams}


    \section{Amazon Data Firehose}


    \section{Amazon Managed Service for Apache Flink}


    \section{Streaming Architectures}


    \section{Amazon MSK}


    \section{AWS Batch}


    \section{Amazon EMR}


    \section{Running Jobs on AWS}


    \section{AWS Glue}


    \section{Redshift}


    \section{Amazon DocumentDB}


    \section{Amazon Timestream}


    \section{Amazon Athena}


    \section{Amazon QuickSight}


    \section{Big Data Architecture}


    \chapter{Monitoring}


    \section{CloudWatch}


    \section{CloudWatch Logs}


    \section{Amazon EventBridge}


    \section{X-Ray}


    \section{AWS Personal Health Dashboard}


    \chapter{Deployment and Instance Management}


    \section{Elastic Beanstalk}


    \section{CodeDeploy}


    \section{CloudFormation}


    \section{Service Catalog}


    \section{SAM - Serverless Application Model}


    \section{AWS CDK - Cloud Development Kit}


    \section{AWS Systems Manger - SSM}


    \section{AWS Cloud Map}


    \chapter{Cost Control}


    \section{Cost Allocation Tags}

    \begin{itemize}
        \item With Tags we can track resources that relate to each other
        \item With Cost Allocation Tags we can enable detailed costing reports
        \item Just like Tags, but they show up as columns in Reports
        \item AWS Generated Cost Allocation Tags
        \begin{itemize}
            \item Automatically applied to the resource you create
            \item Start with Prefix aws: (e.g. aws: createdBy)
            \item They're not applied to resources created before the activation
        \end{itemize}
        \item User tags
        \begin{itemize}
            \item Defined by the user
            \item Start with Prefix user:
        \end{itemize}
        \item Cost Allocation Tags just appear in the Billing Console
        \item Takes up to 24 hours for the tags to show up in the report
    \end{itemize}


    \section{AWS Tag Editor}
    - Allows you to managed tags of multiple resources at once
    - You can add/update/delete tags
    - Search tagged/untagged resources in all AWS Regions


    \section{Trusted Advisor}
    - No need to install anything - high level AWS account assessment
    - Analyse your AWS accounts and provides recommendation
    - Cost Optimisation
    - Performance
    - Security
    - Fault Tolerance
    - Service Limits
    - Operational Excellence
    - Core Checks and recommendations - all customers
    - Can enable weekly email notification from the console
    - Full Trusted Advisor - Available for Business and Enterprise support plans
    - Ability to set CloudWatch alarms when reaching limits
    - Programmatic Access using AWS support API

    \begin{table}[h!]
        \centering
        \begin{tabular}{|c|c|c|c|c|}
            \hline
            Column1                                  & Basic Support                                                        & Column3 & Column4 & Column5 \\ \hline
            AWS Trusted Advisor Best Practice Checks & 7 Core Checks                                                        & Data3   & Data4   & Data5   \\ \hline
            Enhanced Technical Support               & 24x7 customer service, documentation, whitepapers and support forums  & Data8   & Data9   & Data10  \\ \hline
            Case Severity / Response Times           & Data12                                                               & Data13  & Data14  & Data15  \\ \hline
            Data16                                   & Data17                                                               & Data18  & Data19  & Data20  \\ \hline
        \end{tabular}
        \caption{Example Table with 4 Rows and 5 Columns}
        \label{tab:example_table}
    \end{table}

    \subsubsection{Trusted Advisor - Good to Know}
    - Can check if an S3 bucket is made public
    - But cannot check for S3 objects that are public inside of your bucket
    - Use Amazon EventBridge / S3 Events instead / AWS Config Rules

    - Service Limits
    - Limits can only be monitored in Trusted Advisor (cannot be changed)
    - Cases must be created manually in AWS Support Centre to increase limits
    - OR use the AWS Service quotas service


    \section{AWS Service Quotas}
    - Notify you when you're close to a service quota value threshold
    - Create CloudWatch Alarms on the Service Quotas console
    - Example: Lambda concurrent executions
    - Helps you know if you need to request a quota increase or shutdown resources before limit is reached
    %Diagram


    \section{EC2 Launch Types and Savings Plans}
    - On Demand Instances - short workload, predictable pricing, reliable.
    - Spot Instances - short workloads for check, can lose instances (not reliable)
    - Reserved: (Minimum 1 year)
    - Reserved Instances - long workloads
    - Convertible Reserved Instances - long workloads with flexible instances
    - Dedicated Instances: no other customers will share you hardware
    - Dedicated Hosts: book an entire physical server, control instance placement
    - Great for software licenses that operate at the core, or socket level
    - Can define host affinity so that instance reboots are kept on the same host

    \subsection{AWS Savings Plan}
    - New pricing model to get a discount based on long-term usage
    - Commit to a certain type of usage: ex \$10 per hour for 1 to 3 years
    - Any usage beyond the savings plan is billed at the on-demand price

    - EC2 Instance Savings plan (72\% - same discount as Standard RIs)
    - Select instance family and locked to a specific region
    - Flexible across size, OS (Windows to Linux) tenancy. (dedicated or default)
    - Compute Savings Plan
    - Ability to move between instance family, region, compute type and OS and tenancy
    - SageMaker Savings plan (up to 64\% off)


    \section{S3 Cost Savings}

    \subsection{S3 Storage Classes}
    - Amazon S3 Standard - General Purpose
    - Amazon S3 Standard-Infrequent Access (IA)
    - Amazon S3 One Zone-Infrequent Access
    - Amazon S3 Glacier Instant Retrieval
    - Amazon S3 Glacier Flexible Retrieval
    - Amazon S3 Glacier Deep Archive
    - Amazon S3 Intelligent Tiering

    Can move between classes manually or using S3 lifecycle configurations

    \subsection{S3 - Other Cost Savings}
    - S3 Lifecycle Rules: transition objects between tiers
    - Compress Objects - to save space
    - S3 Requester Pays:
    - In general, bucket owners pay for all Amazon S3 storage and data transfer costs associated with their bucket
    - With Requester Pays buckets, the requester instead of the bucket owner pays the cost of the request and the data downloaded from the bucket
    - The bucket owner always pays the cost of storing data
    - Helpful when you want to share large datasets with other accounts
    - If an IAM role is assumed the owner account of that role pays for the request


    \section{S3 Storage Classes - Reminder}

    \subsection{S3 Storage Classes}
    - Amazon S3 Standard - General Purpose
    - Amazon S3 Standard-Infrequent Access (IA)
    - Amazon S3 One Zone-Infrequent Access
    - Amazon S3 Glacier Instant Retrieval
    - Amazon S3 Glacier Flexible Retrieval
    - Amazon S3 Glacier Deep Archive
    - Amazon S3 Intelligent Tiering

    Can move between classes manually or using S3 lifecycle configurations

    \subsection{S3 Durability and Availalbility}
    - Durability:
    - High Durability (99.9999999999, 11 nines) of objects across multiple AZ
    - If you store 10,000,000 object with Amazon S3, you can on average expect to incur a loss of a single object once every 10,000 years (nice....but basic math)
    - Same for all storage classes

    - Availability
    - Measures how readily available a service is
    - Varies depending on storage class
    - Example: S3 standard has 99.99\% availability = not available 53 minutes a year

    S3 Standard - General Purpose
    - 99.99\% Availability
    - Used for frequently accessed data
    - Low latency and high throughput
    - Sustain 2 concurrent facility faiures
    - Use Cases: Big Data analytics, mobile and gaming applications, content and distribution

    S3 Storage Classes - Infrequent Access
    - For data that is less frequently accessed, but requires rapid access when needed
    - Lower cost than S3 standard

    Amazon S3 Standard-Infrequent Access (S3 Standard-IA)
    - 99.9\% Availability
    - Use cases: Disaster Recovery, backups

    Amazon S3 One Zone-Infrequent Access (S3 One Zone-IA)
    - High durability (99.999999999) in a single AZ; data lost when AZ is destroyed
    - 99.5\% Availability
    - Use Cases: Storing secondary backup copies of on-premise data or data you can recreate

    Amazon S3 Glacier Storage Classes
    - Low-cost object storage meant for archiving / backup
    - Pricing: price for storage + object retrieval cost

    - Amazon S3 Glacier Instant Retrieval
    - Millisecond retrieval, great for data accessed once a quarter
    - Minimum storage duration of 90 days
    - Amazon S3 Glacier Flexible Retrieval (Formerly Amazon S3 Glacier)
    - Expedited (1 to 5 minutes), Standard (3 to 5 hours), Bulk (5 to 12 hours) - free
    - Minimum storage duration of 90 days
    - Amazon S3 Glacier Deep Archive - for long term storage
    - Standard (12 hours), Bulk (48 Hours)
    - Minimum Storage duration of 180 days

    \subsection{S3 Intelligent Tiering}
    - Small monthly monitoring and auto-tiering fee
    - Moves objects automatically between Access Tiers based on usage
    - There are no retrieval charges in S3 Intelligent-Tiering

    - Frequent Access tier (automatic): default tier
    - Infrequent Access Tier (automatic): objects not accessed for 30 days
    - Archive Instant Access tier (automatic): objects not accessed for 90 days
    - Archive Access tier (optional): configurable from 90 days to 700+ days
    - Deep Archive Access tier (optional): config from 180 days to 700+ days

    \subsubsection{S3 Storage Classes Comparision}
    %classic table here

    \subsubsection{S3 Storage Classes - Price Comparison}
    %Examples


    \section{AWS Budgets and Cost Explorer}

    \subsection{AWS Budgets}
    - Create budget and send alarms when costs exceeds the budget
    - 4 Types of budgets: Usage, Cost, Reservation, Savings Plans
    - For Reserved Instances (RI)
    - Track utilisation
    - Supports EC2, ElastiCache, RDS, Redshift
    - Up to 5 SNS notifications per budget
    - Can filter by: Service, Linkedin Account, Tag, Purchase Option, Instance Type, Region, Availability Zone, API Operations, etc......
    - Same options as AWS Cost Explorer
    - 2 budgets are free than \$0.002 / day / budget

    \subsection{Budget Actions}
    - Run actions on your behalf when a budget exceeds a certain cost or usage threshold
    - Supports 3 actions types
    - Applying an IAM Policy to a user, group or IAM role
    - Applying Service Control Policy (SCP) to an OU
    - Stop EC2 or RDS Instances
    - Actions can be executed automatically or require a workflow approval process
    - Reduced unintentional overspending in your account.
    %Add diagram

    \subsection{Centralised Budget Management}
    %Diagram

    \subsection{DeCentralised Budget Management}
    %Diagram

    \subsection{Cost Explorer}
    - Visualise, understand and manage your AWS costs and usage over time
    - Create custom reports that analyse cost and usage data
    - Analyse you data at a high level: total costs and usage across all accounts
    - Or Monthly, hourly, resource level granularity
    - Choose an optimal Savings PLan (to lower prices on your bill)
    - Forcast usage up to 12 months based on previous usage

    %Diagram Cost- Explorer example


    \section{AWS Compute Optimiser}
    - Reduce costs and improve performance by recommending optimal AWS resources for you workloads
    - Helps you choose optimal configurations and right-size your workloads (over/under provisioned)
    - Uses Machine Learning to analyse your resources configurations and their utilisation CloudWatch metrics
    - Supported resources
    - EC2 Instances
    - EC2 Auto Scaling Groups
    - EBS volumes
    - Lambda functions
    - Lower your costs by up to 25\%
    - Recommendations can be exported to S3

    \subsection{Computer Optimiser - CloudWatch Agent}
    - Needed to analyse Memory Utilisation
    - Not needed for CPU, NetworkIn/Out, DiskReadOps, DiskWriteOps
    %Diagram


    \section{EC2 Reserved Instance}
    - Reserved Instances in an AWS Organisation
    - All accounts share the Reserved Instances and Savings Plan
    - The payer account (Management account) of an organisation can turn off Reserved Instance (RI) discount and Savings Plans discount sharing for any accounts in that organisation, including the payer account.
    - Renewal of Reserved Instances
    - You can queue (schedule or reserve ahead of time) you reserved instances
    - To renew a RI, just queue an RI purchase whenever the previous one expires


    \chapter{Migration}


    \section{Cloud Migration Strategies - The 7Rs}


    \section{Storage Gateway}


    \section{Storage Gateway - Advanved Concepts}


    \section{Snow Family}


    \section{Snow Family - Improving Performance}


    \section{AWS DMS - Database Migration Services}


    \section{AWS CART - Cloud Adoption Readiness Tool}


    \section{Disaster Recovery}


    \section{AWS FIS - Fault Injection Simulator}


    \section{VM Migrations Serrvices}


    \section{AWS Migration Evaluator}


    \section{AWS Backup}


    \chapter{VPC}


    \section{VPC - Basiscs}


    \section{VPC - Peering}


    \section{Transit Gateway}


    \section{VPC Endpoints}ó


    \section{VPC Enpoint Policies}


    \section{PrivateLink}


    \section{AWS S2S VPN}


    \section{AWS Client VPN}


    \section{Direct Connect}


    \section{On-Premise Redundant Connections}


    \section{VPC Flow Logs}


    \section{AWS Network Firewall}


    \chapter{Machine Learning}


    \section{Rekognition Overview}
    - Find objects, people, text, scenes in images and videos using ML
    - Facial analysis and facial search to do user verification, people counting.
    - Create a database of "familiar faces" or compare against celebrities
    - Use cases:
    - Labeling
    - Content Moderation
    - Text Detection
    - Face Detection and Analysis (gender, age range, emotions....)
    - Face Search and Verification
    - Celebrity Recognition
    - Pathing (ex: for sports game analysis)

    \subsection{Amazon Rekognition - Content Moderation}
    - Detect content that is inappropriate, unwanted or offensive (image and videos)
    - Used in social media, broadcast media, advertising and e-commerce situation to create a safer user experience
    - Set a Minimum Confidence Threshold for items that will be flagged
    - Flag sensitive content for manual review in Amazon Augmented AI (A2I)
    - Help comply with regulations


    \section{Transcribe Overview}
    - Automatically convert speech to text.
    - Uses a deep learning process called automatic speech recognition (ASR) to convert speech to text quickly and accurately.
    - Automatically remove Personally Identifiable Information (PII) using Redaction.
    - Supports Automatic Language Identification for multilingual audio
    - Use cases:
    - transcribe customer service calls
    - automate closed captioning and subtitling
    - generate metadata for media assets to create a fully searchable archive


    \section{Polly Overview}
    - Turn text into lifelike speech using deep learning
    - Allowing you to create applications that talk

    \subsection{Amazon Polly - Lexicon \& SSML}
    - Customise the pronunciation of words with Pronunciation lexicons
    - Stylized words: St3ph4ne => "Stephane"
    - Acronyms: AWS => "Amazon Web Services"
    - Upload the lexicons and use them in the SynthesizeSpeech operation.
    - Generate speech from plain text or from documents marked up with Speech Synthesis Markup Language (SSML) - enables more customisation
    - Emphasising specific words or phrases.
    - Using phonetic pronunciation.
    - Including breathing sounds, whispering.
    - Using the Newscaster speaking style


    \section{Translate Overview}
    - Natural and accurate language translation
    - Allows you to localise content - such as websites and applications - for international users and to easily translate large volumes of text efficiently.


    \section{Lex + Connect Overview}
    - Amazon Lex: (same technology that powers Alexa)
    - Automatic Speech Recognition (ASR) to convert speech to text
    - Natural Language Understanding to recognise the intent of text, callers
    - Helps build chatbots and call centre bots
    - Amazon Connect:
    - Receive calls, create contact flows, cloud-based virtual contact centre
    - Can integrate with other CRM systems of AWS
    - No upfront payments, 80\% cheaper than traditional contact center solutions \textit{Hmmmm......}


    \section{Comprehend Overview}
    - For Natural Language Processing - NLP
    - Fully managed and serverless service
    - Uses machine learning to find insights and relationships in text
    - Language of the text
    - Extracts key phrases, places, people, brands or events
    - Understands how positive or negative the text is
    - Analyses text using tokenization and parts of speech
    - Automatically organises a collection of text files by topic
    - Sample use cases:
    - Analyse customer interactions (emails) to find what leads to a positive or negative experience
    - Create and groups articles by topics that Comprehend will uncover


    \section{Comprehend Medical Overview}
    - Amazon Comprehend Medical detects and returns useful information in unstructured clinical text:
    - Physician's notes
    - Discharge summaries
    - Test results
    - Case notes
    - Uses NLP to detect Protected Health Information (PHI) - DetectPHI API
    - Store your documents in Amazon S3, analyse real-time data with Kinesis Data Firehose or use Amazon Transcribe to transcribe patient narratives into text that can be analysed by Amazon Comprehend Medical.


    \section{SageMaker Overview}
    - Fully managed service for developers / data scientists to build ML models
    - Typically difficult to do all processes in one place + provision servers


    \section{Kendra Overview}
    - Fully managed document search service powered by Machine Learning
    - Extract answers from within a document (text, pdf, HTML, PowerPoint, MS Word, FAQs)
    - Natural language search capabilities
    - Learn from user interactions / feedback to promote preferred results (Incremental Learning)
    - Ability to manually fine-tune search results (importance of data, freshness, customer)


    \section{Personalise Overview}
    - Fully managed ML-service to build apps with real-time personalised recommendations
    - Example: personalised product recommendations/re-ranking, customised direct marketing
    Example: User bought gardening tools, provide recommendations on the next one to buy.
    - Same technology used by Amazon.com
    - Integrates into existing websites, applications, SMS, email marketing systems.........
    - Implement in days, not months (don't need to build, train and deploy ML solutions)
    - Use cases; retail stores, media and entertainment


    \section{Textract Overview}
    - Automatically extracts text, handwriting and data from any scanned documents using AI and ML
    - Extract data from forms and tables
    - Read and process any type of document (PDFs, images)
    - Use cases:
    - Financial Services (e.g, invoices, financial reports)
    - Healthcare (e.g medical records, insurance claims)
    - Public Sector (e.g tax forms, ID documents, passports)


    \section{Machine Learning Summary}

    \begin{itemize}
        \item Rekognition - face detection, labeling, celebrity recognition
        \item Transcribe - audio to text (ex: subtitles)
        \item Polly - text to audio
        \item Translate - translations
        \item Lex - build conversational bots - chatbots
        \item Comprehend - natural language processing
        \item SageMaker - Machine learning for every developer and data scientist
        \item Kendra - ML-powered search engine
        \item Personalise - real-time personalised recommendations
        \item Textract - detect text and data in documents
    \end{itemize}


    \chapter{Other Services}


    \section{Other Services}


    \section{IMPORTANT: CodeCommit Discontinuation}


    \section{CI/CD}


    \section{Amazon CodeGuru}


    \section{Alexa for Business, Lex and Connect}


    \section{Kinesis Video Streams}


    \section{AWS WorkSpaces}


    \section{Amazon AppStream 2.0}


    \section{AWS Device Farm}


    \section{Amazon Macie}


    \section{Amazon SES}


    \section{Amazon Pinpoint}


    \section{EC2 Image Builder}


    \section{Other Services Summary}


    \chapter{Exam Preparation}

\end{document}